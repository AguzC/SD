\documentclass{article}
\usepackage{graphicx} % Required for inserting images
\usepackage{amssymb}
\usepackage{amsfonts}

\title{Guia 2 Algo2}
\author{Agustin C}
\newcommand{\italica}[1]{\textit{#1}}
\newcommand{\conjuntito}[1]{\langle #1 \rangle}
\newcommand{\enteros}[0]{\mathbb{Z}}
\newcommand{\implicaL}[0]{\rightarrow_L}
\newcommand{\implica}[0]{\rightarrow}
\newcommand{\sumatoria}[2]{\sum_{#1}^{#2}}
\newcommand{\sisolosi}[0]{\leftrightarrow}


\begin{document}

%comentarios
\maketitle
\section*{4)}
\subsection*{4.a)}
pred \italica{esPrefijo} (s, t: seq  $ \conjuntito{\enteros}$) 
\{ 
$ \\ |s| \leq |t| \land_L (\forall i:\enteros) 
(0 \leq i < |s| \implicaL (s[i] = t[i])) \\ \}$
\subsection*{4.b)}
pred \italica{estaOrdenada} (s: seq  
$\conjuntito{\enteros}) \{$

$(\forall i:\enteros) (0 \leq i < |s| \implicaL s[i] \leq s[i + 1] ) \\ \}$
\subsection*{4.c)}
pred \italica{hayUnoParQueDivideAlResto} $ (s: seq$
$\conjuntito{\enteros}) \{ $ 

$(\exists i:\enteros) (0 \leq i < |s| \land_L \italica{esPar }(s[i]))
\land_L ((\forall j: \enteros)((0 \leq j < |s|) \implicaL (s[j]$  mod  $ s[i] = 0)))$
$ \\ \} $

\subsection*{4.d)}
pred \italica{enTresPartes} $ (s: seq$
$\conjuntito{\enteros}) \{ $ 

$ estaOrdenada(s) \land soloCerosUnosDos(s) $ \\ $\} \\$
pred \italica{soloCerosUnosDos} $ (s: seq$
$\conjuntito{\enteros}) \{ $ 

$(\forall i: \enteros) ((0 \leq i < |s| \implicaL 
(s[i] = 0 \vee s[i] = 1 \vee s[i] = 2)) \\ \}$

\section*{5)}

\subsection*{5.a)}

\italica{apariciones}(e) $= \sumatoria{i = 0}{|s|-1}$ IfThenElse(s[i] = e,1,0)

\subsection*{5.b)}

\italica{totalPosicionImpar}(s) $= \sumatoria{i = 0}{|s|-1} ifThenElse(\neg \italica{esPar}(i),s[i],0)$

\subsection*{5.c)}
\italica{sumaMayor0}$\sumatoria{i = 0}{|s|-1} IfThenElse(s[i]>0,s[i],0)$
\subsection*{5.d)} 
\italica{sumaInversoNoCero} $= \sumatoria{i=0}{|s|-1} IfThenElse(s[i]\neq 0,\frac{1}{s[1]},0)$

\section*{6)} 
\subsection*{6.a)} 

\italica{progresionGeometricaFactor2}: Indica si la secuencia l representa una progresion geometrica factor 2. Es decir, si cada
elemento de la secuencia es el doble del elemento anterior. $\\$


El problema de la especificacion es que se puede indefinir para i = 0, porque no hay s[-1],
asi que hay que cambiar el rango de i:$\\$

asegura $\{res = True \sisolosi (\forall i:\enteros)((1 \leq i < |l|) \implicaL l[i] = 2*l[i-1])$

\subsection*{6.b)}
\italica{minimo} devuelve el menor elemento de l. $\\$

La lista vacia lo indefine y el res que devuelve no esta aclarado si es de la lista 
o no, asique puede devolver cualquier numero que sea menor a todos los elementos de 
la lista.

requiere $\{ (|l| > 0) \} $


asegura $\{ (res \in l)  \land (\forall y:\enteros) ( y \in l \implicaL 
y \geq res) \}$

\section*{7)} Dar todas las soluciones posiblesa las entradas dadas.
\subsection*{7.a)}

I) Devuelve 3 $\\$
II) Devuelve 0 o 3, son igual de validos $\\$
III) Devuelve cualquiera entre 0 a 5, son igual de validos.

\subsection*{7.b)}
I) Devuelve 3 $\\$
II) Devuelve 0 $\\$
III) Devuelve 0

\subsection*{7.c)}
Para listas sin numeros repetidos.

\section*{8)}

\subsection*{8.a)}
Por el AND entre las condiciones, es una contradiccion, nunca se va 
a cumplir que $(a < 0) \land (a \geq 0)$.
\subsection*{8.b)}
Es correcto, asi se escriben los If.
\subsection*{8.c)}
Es incorrecto porque los valores de verdad del implica dicen que si la primera
condicion no se cumple, entonces la implicacion siempre es verdadera, esa 
formula es una tautologia.
\subsection*{8.d)}
Ta correcta pues.

\section*{9}
\subsection*{9.a)}
Con $x = 3$ el algoritmo devuelve 9, que cumple la postcondicion.
\subsection*{9.b)}
Devuelve 0.25, 1, 0.04 y 49 respectivamente.
\subsection*{8.c)}
requiere $\{ |x| > 1 \}$

\section*{10)}
\subsection*{10.a)}
P3 es mas fuerte que P2 y P1, P3$\implica$P2, P3$\implica$P1 y P1$\implica$P2
\subsection*{10.b)}
Q3 es mas fuerte que Q2 y Q1, Q3$\implica$Q2, Q3$\implica$Q1 y Q1$\implica$Q2
\subsection*{9.c)}
Supongo que pide una implementacion porque el punto dice que cumpla esa 
especificacion....., asi que diria que un return X*X y otro return (X*X) + 1 
cumplen eso....(?.
\subsection*{9.d)}
a) Ya que la precondicion es mas fuerte, cumple igual.  $\\$
b) Ya que la postcondicion es mas debil, no sabemos que pasa con los numeros del 1 al 10 $\\$
c) Ya que la postcondicion es mas debil, todos los resultados cumplen. $\\$
d) Ya que la postcondicion es mas fuerte, no cumple. $\\$
e) Ya que la pre condicion es mas fuerte y la post mas debil, todos cumplen.$\\$
f) Ya que la pre condicion es mas debil, los numeros del 1 al 10 no tienen una 
solucion definida, y la postcondiciones mas fuerte, asi que no va a cumplir tampoco.

\subsection{9.e)}
Hacer la precondicion mas fuerte para garantizar criterios minimos y que no se rompa, y 
la post mas debil asi todo cumple.... 

hola




\end{document}
